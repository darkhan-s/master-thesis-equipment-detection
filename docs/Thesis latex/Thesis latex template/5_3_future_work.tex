\subsection{Directions for future work}
Inspired by Oza et al. \cite{Oza2021} and by the \nameref{objective}, this thesis studied multiple research directions in domain adaptation. Oza et al. summarized a few scenarios, where the cross-domain object detection was not researched as extensively. A few of these scenarios overlap with the thesis goals and these will be discussed in the following section. 

\subsubsection{Other applications and real world datasets}
As it was discussed in \nameref{mainExperiments} section, common DA approaches in object detection address either autonomous navigation, or common day-to-day datasets and to the best of the author's knowledge, no DA work has been conducted on  industrial datasets. Additionally, many of these datasets are barely suitable to reflect the domain gap between the real world and the simulations. This thesis attempts to reduce the knowledge gap by utilizing a relatively industrial dataset. However, the target dataset proposed in the \nameref{datasets} section was collected in a typical household setup, where the environment has a stable lighting and background. Therefore, for future work, it is advised to experiment with images from a real industrial plant. \todo{Modify if we manage to obtain real equipment pics} 

\subsubsection{Other detectors}
As it could be noted from the \nameref{DAobj} section, Faster-RCNN is a de-facto object detector framework used in such problems. Oza et al. \cite{Oza2021} suggested to experiment with single-stage object detectors such as YOLO \cite{Redmon2015a}, SSD \cite{Liu2015}, FCOS \cite{Tian2019} and DETR \cite{Carion2020}. In this thesis, two-stage Faster-RCNN was selected due to its simple plug-and-play compatibility with Detectron2 \cite{wu2019Detectron2} and with the existing DA methods. Additionally, the robustness that single-stage detectors offer is not as critical in this thesis. However, there is a potential of extending the application to video-stream and identify the objects in real-time. As it was discussed in the \nameref{obj_detection_section} section, single-stage detectors offer a significant speed improvement over Faster-RCNN. Upon analyzing the feasibility of implementing other detectors in Detectron2 \cite{wu2019Detectron2} framework, it was identified that FCOS network has already been adapted for Detectron2. Hence, it is believed that among the detectors listed, the FCOS detector will be the most compatible with the setup proposed in Figure \ref{mymodel}. 


\subsubsection{Improved scheduler}
This thesis has only evaluated the cosine annealing scheduler without restarts. One possible way to improve the results presented in the \nameref{scheduler_section} could be achieved by implementing the restart cycles introduced in the original paper by Loschilov et al. \cite{Loshchilov2016}.   


\subsubsection{Imbalanced classes}
Although the classes were distributed fairly well in the source dataset, huge class imbalance was observed in the target T-LESS dataset, as it can be noted from Figure \ref{tless_distribution_real}. The models 1 and 4 over-represented and are the largest in the target dataset by a tremendous margin. This could be a potential reason for the performance drop as the disproportionately highly represented classes shifted the alignment of the domains in their favor. Oza et al. proposed re-weighting the classes based on their frequency. However, as the target domain does not contain any labels, it becomes a challenging task to track the frequency \cite{Oza2021}. Luckily, the proposed in the \nameref{mainExperiments} section cross-domain adaptation method leverages pseudo-labels. Therefore, re-weighting the classes can be carried out in stages once pseudo-labeling accuracy improves. 

On the other hand, it is possible to benefit from the class imbalance. Section \ref{cont_learning_results} discusses objects that are harder to detect. Instead of trying to re-weight imbalanced classes, one could provide a higher number of image samples for the objects that can be classified as "harder to detect", while providing a regular number of samples for the objects that are easier to detect.

\subsubsection{Multi-source adaptation}
Another practical problem formulated by Oza et al. is that often the images can be collected from multiple domains. In the scope of the thesis, this could imply collecting datasets from multiple plants with varying environmental conditions, such as rainy weather, snow, daylight and nighttime. These conditions will dramatically impact the detection results. Therefore, more experiments should be carried out before releasing the application into use. One suggested approach to this problem could be carried out by using even more sophisticated augmentations to simulate such environments \cite{imgaug}. 


\subsubsection{Continual learning}
Although the \nameref{cont_learning_section} section addressed the catastrophic forgetting problem in a scalable setup, the experiments were only conducted with well-established and possibly outdated methods. Moreover, according to Parisi et al. \cite{Parisi2018}, current state-of-the-art continual learning models are still far from being able to learn new tasks as efficiently as biological systems and to this day it remains a challenging problem. Additionally, the method proposed raise an additional constraint. Although the model is able to adapt to the dynamically expanding dataset, the model will scale linearly as new classes are added. Within this thesis, the model was tested on up to 30 classes and the resulted model files consumed nearly 2GB of storage and a few linear layers appended will not significantly affect the already high file size. Nevertheless, for the sake of optimization and as new continual learning solutions emerge, it is important to evaluate them in the given setup and explore new possible solutions. 




\clearpage