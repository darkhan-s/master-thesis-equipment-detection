\section{Introduction}

%% Leave page number of the first page empty
%% 
\thispagestyle{empty}
\subsection{Problem statement}
In recent years, computer vision algorithms have received much attention due to their potential applications in a vast variety of fields, including security monitoring \cite{Awalgaonkar2020}, medicine \cite{9689485}, and self-driving vehicles \cite{Janai2017, Shan2018}. However, although computer vision has been integrated into industrial applications (e.g., safety and process monitoring) \cite{Awalgaonkar2020, Banf2022}, less research has addressed the issue of industrial equipment detection \cite{Wu2022, MALBURG2021581, Kim2020}. 

As industrial plants are typically hundreds of meters long, it often becomes frustrating to identify equipment parts for maintenance or replacement. Ore processing plants treat several hundred tons of ore per hour, and the production capacity is constant. Therefore, it is often difficult to properly identify the equipment within a list of thousands of parts in a medium- to large-scale plant.

This work has been commissioned by Metso Outotec Oyj.  Metso Outotec offers digital solutions that enable customers to automate their processes in the mining, aggregates and metals industries. In order to ensure that these processes operate as smoothly as possible, it is important to optimize them at all stages of production. Recently, Metso Outotec has successfully applied computer vision in applications for identifying foreign objects in crushing processes \cite{metso_outotec_2022}, for detecting defects in copper  molds \cite{metso_outotec_2022_2}, as well as for recognizing froth characteristics in flotation cells \cite{metso_outotec_2022_1}, to name a few. However, the company has not yet attempted to apply computer vision for facilitating maintenance. For these reasons, Metso Outotec has requested to investigate the feasibility of applying state-of-the-art computer vision algorithms to equipment recognition. 

Even though various methods have been implemented for detection of objects in a countless number of fields \cite{ima, Liu2015, He2017, Redmon2015a, Zhang2021b, Tian2019}, these methods heavily rely on extensive data collection and training of models in order to accurately identify objects. Moreover, complications arise, as it is often not possible to collect huge amounts of training images from industrial environments due to privacy and confidentiality issues. Luckily, for this project, the images can rather easily be collected from a 3D simulator model of a gold refining plant. However, using the rendered images from a 3D simulator limits the accuracy of the models, as such models do not perform as well on real images due to the domain shift phenomenon \cite{Ganin2015}, which occurs when the environmental conditions change at the time of capturing training and test images. 

Hence, this thesis proposes a cross-domain object detection approach as a solution to automatically localize and identify the equipment in a large industrial environment in order to minimize the delay in production arising as a result of manual identification. 

%https://www.mogroup.com/corporate/
%https://www.mogroup.com/corporate/about-us/

\clearpage

\subsection{Thesis objective}
\label{objective} 
The main goal of this thesis is to identify a suitable state-of-the-art object detection technique and to enhance its performance on the custom equipment dataset. The proposed method should be able to identify an object in a real image given a labeled dataset of rendered images from a 3D model and a smaller unlabeled dataset of real images. Additionally, the developed method should provide a solution for optimizing the laborious process of data collection and labeling. Furthermore, the produced model should address the cases when new objects are continuously added to the dataset.  Such optimization is important not only because training the model from the scratch is a time-demanding process, but also because large plants contain thousands of objects, thus making scalability a critical requirement. Finally, a minimal proof-of-concept application should be prepared to demonstrate the performance of the proposed detection technique.  

\subsection{Methodology}
In order to accomplish these objectives, the thesis will first explore state-of-the-art object detection frameworks, libraries and algorithms. Similarly, domain adaptation algorithms will be analyzed in an object detection setup. The most suitable methodologies will then be then selected to be used in a novel cross-domain object detection model.  

In order to circumvent regulations regarding accessibility and confidentiality, the dataset utilized for training the model in the experimental scenario will be based on the T-LESS open-source dataset \cite{hodan2017tless}. Since the dataset was originally intended for pose estimation in 3D models, it will be converted into formats appropriate for the proposed object detection algorithms.

To achieve higher performance in object detection, the domain shift phenomenon will be addressed using the Adaptive Teacher \cite{Li2021} algorithm for cross-domain object detection, which in turn uses the Faster-RCNN \cite{ima} implementation as a detector base in the Detectron2 \cite{wu2019Detectron2} framework.  The thesis will contribute to current knowledge by introducing an instance-level domain classifier appended to the base network of the Adaptive Teacher algorithm, as suggested by Chen et al. \cite{Chen2018}. Additionally, the study will evaluate the feasibility of other strategies, such as continual learning \cite{Parisi2018}, to further enhance scalability of the model. The model will then be integrated into a prototype web application for demonstration purposes. The produced model will be trained on rendered data from 3D models and evaluated on real images using mean average precision metrics. Finally, the proposed method will be evaluated using one equipment item from a real plant operated by a Metso Outotec client.
\todo{Verify if obtaining real equipment data is feasible}

\clearpage

\subsection{Scope}

The thesis will be limited to proposing a minimal proof-of-concept solution based on analyzing and combining different components of existing state-of-the-art models. In addition, this solution will be wrapped in a prototype web application. However, preparing an actual real-life dataset and implementing the solution for a real plant remains outside the scope of this study due to the time constraints. Although the proposed method attempts to optimize the data collection and labeling process, this will in practice require many months before the dataset and the model based on real data would be ready for use. 

For the user interface, a prototype will be provided in order to showcase the performance of the model. However, the thesis will primarily focus on deep learning algorithms rather than methods to deploy a model. For this reason, the prototype will only offer basic functionality. Finally, due to time constraints, a video-compatible model will remain outside the scope of this work. 



\subsection{Structure of the thesis}
The rest of this thesis is divided into four chapters. Chapter 2 reviews the common terminology on deep learning, image classification and object detection. Additionally, the chapter discusses the state-of-the art research in industrial object detectors. Furthermore, the latest domain adaptation and cross-domain object detection techniques are covered in greater detail. Finally, the topic of continual learning is reviewed. Chapter 3 defines the dataset, the evaluation metrics used, as well as evaluates the standard object detectors and several cross-domain object detection algorithms. Finally, it proposes a novel architecture, which addresses both domain adaptation and continual learning. Chapter 4 evaluates the proposed solution and compares the results to other methods using average precision metrics. Chapter 5 summarizes this work by discussing the proposed architecture and suggesting directions for future work.

%% In a thesis, every section starts a new page, hence \clearpage
\clearpage